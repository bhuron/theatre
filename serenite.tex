\documentclass[a4paper,12pt]{article}
\usepackage[utf8]{inputenc}
\usepackage[T1]{fontenc}
\usepackage[french]{babel}
\usepackage{ebgaramond}
\usepackage{dramatist}
\usepackage{dramatist}
\usepackage{geometry}
\geometry{margin=3cm}
\linespread{1.15}
\renewcommand{\speaksfont}{\bfseries\scshape}
\setlength{\speechskip}{1em}
\frenchbsetup{og=«,fg=»}
\title{Sérénité modèle confort}
\author{Marc Pheline et Odile Clair}
\date{}

\begin{document}
\maketitle

\Character[Un homme]{LUI}{lui}
\Character[Une femme]{ELLE}{elle}

\begin{drama}

\luispeaks Minou...

\ellespeaks ...

\luispeaks Tu dors ?

\ellespeaks \direct{Grognement.}

\luispeaks Tu dors, oui ou non ?

\ellespeaks OUI!

\luispeaks Déjà?

\ellespeaks \direct{Soupir.}

\luispeaks Excuse-moi. Je ne pensais pas...

\ellespeaks S'il te plaît, minou...

\luispeaks Non, sans rire, tu dormais vraiment ?

\ellespeaks \direct{Très faiblement.} Si...

\luispeaks C'est drôle comme tu t'endors vite, toi, hum?

\ellespeaks ...

\luispeaks Tu en as de la chance. Tu plonges tout de suite, un vrai bébé. Tu es réglée comme du papier à musique.

\ellespeaks \direct{Grognement.}

\luispeaks Les gens fonctionnent complètement différemment les uns des autres au niveau du déclic-dodo. Tiens, toi et moi par exemple, rien à voir! Chacun fait sa petite cuisine intérieure pour atteindre l'état de grâce. Toi, ce serait plutôt du précuit. Ha! Ha! Moi, il faut que ça mijote longtemps. Tant que je ne me suis pas retourné plusieurs fois... Tu m'écoutes ?

\ellespeaks \direct{Grognement.}

\luispeaks Qu'est-ce que tu as dit?

\ellespeaks Rien...

\luispeaks Mais si, j'ai bien entendu, tu as marmonné quelque chose dans ta barbe.

\ellespeaks Laisse-moi dormir !

\luispeaks C'était pas ça.

\ellespeaks Peut-être, mais maintenant, j'aimerais bien dormir, si tu permets.

\luispeaks Mais, ma chérie, je veux bien que tu dormes, mais tu m'as parlé, et je n'ai pas compris... Je m'intéresse à ce que tu dis, de temps en temps...

\ellespeaks Fous-moi la paix ! C'que tu es pénible!

\luispeaks Inutile d'être agressive! J'ai le malheur de ne pas entendre ce que madame dit dans son demi-sommeil parce que madame DORT pendant que j'ai des soucis, je me renseigne, et je suis « pénible ! Tu me reprocheras de ne pas écouter quand je parl... Euh... Quand TU parles...

\ellespeaks La barbe ! J'en ai marre que tu me réveilles chaque fois que tu as tes états d'âme ! Tu sais très bien que quand on me réveille dans mon premier sommeil, je ne peux plus me rendormir, t'es pas sympa...

\luispeaks Ça y est! Une fois de plus, c'est moi l'emmerdeur !

\ellespeaks Mais minou, ne me dis pas que...

\luispeaks Non, je ne dis rien. Moi, quand on me traite comme ça, je me tais. Bonsoir !

\ellespeaks Écoute, personne ne t'a traité d'emmerdeur !

\luispeaks Tiens donc ! A peine!

\ellespeaks Pas du tout. C'est toi qui l'as dit. Nuance !

\luispeaks Ben, voyons! Facile, de retourner les situations !

\ellespeaks Retourner les situat...!

\luispeaks Allez, tais-toi, va, dors, puisque tu y tiens tellement ! Mais n'essaye pas de te disculper, tu es vraiment grotesque. Bonne nuit !

\ellespeaks Ah non! Tu ne t'en tireras pas comme ça ! Je ne t'ai pas traité d'emmerdeur, et tu ne t'endormiras pas avant de l'avoir reconnu... Tu entends?

\luispeaks Chut! L'emmerdeur DORT.

\ellespeaks Tu es vraiment le...

\luispeaks Dernier des salauds, oui, je sais. Et c'est pour ça que tu m'aimes. Maintenant arrête, ou ça va tourner au vinaigre. Allez, dodo.

\ellespeaks Je te déteste.

\luispeaks Mais oui. Jusqu'à demain matin. Dors, ça passera plus vite.

\ellespeaks Si je veux.

\luispeaks Tu voulais, non ? Alors, dors !

\ellespeaks Tu n'as pas d'ordres à me donner!

\luispeaks Fais ce que tu veux. Moi je dors.

\ellespeaks Moi aussi. Là !

\luispeaks Ah. Bonne nuit.

\ellespeaks Bonne nuit.

\StageDir{Temps.}

\StageDir{Elle soupire.}

\luispeaks Chut !...

\StageDir{Elle soupire encore.}

\luispeaks Rhôôô... Tu peux arrêter de gigoter?

\ellespeaks Y a un ressort.

\luispeaks Penses-tu...

\ellespeaks Mais si, y a un ressort!

\luispeaks C'est bien fait. Tu n'as pas voulu payer quatre fois moins cher pour un matelas-mousse.

\ellespeaks Peut-être, mais il y a un ressort.

\luispeaks Pas étonnant, y en a même plus d'un, y en a au moins... six cents !

\ellespeaks N'importe quoi! Tu penses qu'ils se sont donné la peine d'en mettre autant? Tu rêves ! Doit y en avoir une vingtaine, à peine, vingt-cinq à tout casser.

\luispeaks Tu crois ça ? Tu vas voir.

\StageDir{Il sort.}

\ellespeaks Mais... Ne fouille pas dans mes affaires. Qu'est-ce que tu cherches ?

\luispeaks \direct{En coulisse.} La notice.

\ellespeaks Oh, mais je m'en fous, moi... Vingt-cinq, six cents ou trois millions de ressorts... Y en a qu'un de travers, et comme par hasard, c'est moi qui l'ai.

\luispeaks \direct{Rentrant.} Ah, voilà: « Composition générale du matelas Sérénité Modèle Confort. Type Luxe. »

\ellespeaks Tu parles d'un confort...

\luispeaks Euh... « Tissu de recouvrement »... Non, c'est pas ça... « Face été, face hiv... » Non plus... Ah ! « Support de base: environ cent deux ressorts biconiques de suspension thermostabilisés, ET environ quatre-vingt-sept ressorts de traction en continu, soit environ cent quatre-vingt-neuf ressorts traités AU MÈTRE CARRÉ ! »

\ellespeaks Dont un qui me rentre dans les reins !

\luispeaks Ça nous fait... Attends voir... Cent quarante sur deux... deux mètres carrés quatre-vingts... multipliés par cent quatre-vingt-neuf... huit fois neuf... deux et je retiens sept... Hum... huit fois huit... et sept...

\ellespeaks Cinq cent vingt-neuf virgule deux.

\luispeaks Pff! Sûrement pas ! Réfléchis, il ne peut pas y avoir de virgule, les ressorts, ça ne se découpe pas en rondelles !

\ellespeaks Jusqu'à nouvel ordre, cent quatre-vingt-neuf multiplié par deux virgule huit, ça fait cinq cent vingt-neuf virgule deux. Oui monsieur !

\luispeaks Hé ben voilà! C'est la virgule qui te gêne !

\ellespeaks Ha! ha ! ha ! Tordant! Tu veux ma place ?

\luispeaks Dis donc, tu vas arrêter de me casser les pieds ? J'en ai eu pour plus de trois mille cinq cents balles de sérénité, avec ce plumard, que TU as choisi, permets-moi de te le rappeler, et j'aimerais bien profiter tranquillement de MON investissement.

\ellespeaks Très bien. Éteins.

\luispeaks Non, je lis...

\ellespeaks \direct{Soupir exaspéré.}

\luispeaks Tiens ! Il est garanti dix ans...

\StageDir{Elle se cache la tête sous l'oreiller.}

\luispeaks Oh, ce que c'est bien foutu, ces trucs-là! Tu te rends compte, pour, l'enrobage de climatisation à élasticités superposées, il y a, sur CHAQUE face, une toile, un feutre, une ouate, en textiles, textiles au pluriel, s'il vous plaît, et une plaque isolante de synthèse !

\ellespeaks \direct{Sous l'oreiller :} N'empêche, il est dur!

\luispeaks C'est excellent pour la colonne... Qu'est-ce qu'ils ont encadré, là? Tu peux lire, s'il te plaît, moi, sans mes lunettes... (Elle ne réagit pas.) « Attention, les qualités et densités des composants de base varient selon la gamme qualitative de nos produits. Nous, en tout cas, on a tapé dans ce qu'il y a de mieux...

\ellespeaks Mais alors, question densité des composants, on s'est carrément fait couillonner!

\luispeaks Tu n'es jamais contente ! Il est quand même homologué par le SNL !

\ellespeaks Ça nous fait une belle jambe! Quand bien même il serait recommandé par la SNCF ou par l'UNESCO...

\luispeaks Déconne pas, c'est sérieux, c'est le Syndicat national de la literie.

\ellespeaks Ah... Dans ce cas, j'ai rien à dire...

\luispeaks Tant mieux. Sur ce, je crois que je vais bien dormir...

\ellespeaks C'est ça. Eteins. (Il éteint.) Merci.

\luispeaks De rien. Tu me fais un petit bisou?

\ellespeaks Rhôôô! Écoute!

\luispeaks Allez...

\ellespeaks \direct{Soupir excédé. Bisou.}

\StageDir{Temps.}

\ellespeaks Minou!

\luispeaks Chut...

\ellespeaks Minou, rallume !

\luispeaks Ah non, ça suffit !

\ellespeaks Arrête, rallume, y a une bête ! J'ai senti quelque chose! Sur la figure!

\luispeaks \direct{II rallume.} Quoi, quoi, qu'est-ce que...

\ellespeaks Là!

\luispeaks Oh! Qu'elle est belle ! Bonsoir petite araignée ! L'est mignonne la p'tite araignée.

\ellespeaks Chasse-la, mais enfin, chasse-la !

\luispeaks Allez, faut aller faire son dodo, maintenant. Regarde, elle agite une de ses huit gracieuses petites pattes pour dire bonsoir, tu as vu comme elles sont velues, en voilà une araignée bien élevée !

\ellespeaks Tue-la!

\luispeaks Ça va pas ?! Elle est si jolie, et elle ne ferait pas de mal à une mouche, enfin si, justement, mais c'est pour ça qu'il ne faut pas la tuer, c'est très utile les araignées.

\ellespeaks Ou tu la tues, ou je crie !

\luispeaks C'est déjà fait.

\ellespeaks Alors je vais recommencer ! Je t'en supplie, je ne supporte pas ça !

\luispeaks Je ne vois vraiment aucune raison défendable de me rendre coupable d'un meurtre. Tu ne me feras pas participer à ce lâche et traditionnel génocide, je préfère laisser aux hystériques de tous poils le soin d'accomplir ce rite absurde.

\ellespeaks J'ai PEUR, je te dis ! Ces bestioles, ça me rend...

\luispeaks Ces « bestioles » appartiennent à l'ordre des aranéides, lui-même inclus dans la classe des arachnides, laquelle comprend par ailleurs, outre les aranéides, les scorpionides et les acariens, les poux quoi. La caractéristique majeure distinguant l'araignée des autres groupes de chélifères réside dans cet abdomen non segmenté que tu peux observer là, très nettement séparé du céphalothorax.

\ellespeaks Arrête, je ne me sens pas bien.

\luispeaks Cela dit, dans la sous-classe des araignées, si les représentants des trois ordres sont également venimeux, puisque également pourvus de crochets, deux seulement présentent un danger mortel...

\ellespeaks Elle a bougé !

\luispeaks Mais le troisième, celui des aranéomorphes, qui nous intéresse plus particulièrement, car c'est manifestement à celui-ci qu'appartient ce superbe spécimen, femelle si je ne m'abuse, le troisième donc, est à la fois le plus répandu et le plus inoffensif. De plus, araignée du soir espoir, tu vois bien que tu n'as rien à...

\ellespeaks \direct{En pleurant.} Minou... Je... Tu... Je te préviens, si tu ne tues pas immédiatement cet arachnoïde, je...

\luispeaks Arachnide, ma chérie, arachnide.

\ellespeaks Cette... saloperie, tu risques de le regretter!

\luispeaks Sais-tu que d'éminentes personnalités du monde scientifique s'accordent à reconnaître que certaines personnes à sensibilité délicate focalisent volontiers leurs fantasmes sur un animal quelconque, les plus répandus se trouvant être l'araignée, le serpent, le rat, les rongeurs en général. Ainsi, une pauvre créature innocente concentre sur elle l'essentiel des angoisses, et donc de l'agressivité du sujet.

\StageDir{Elle se lève et sort.}

\luispeaks Où vas-tu ?

\ellespeaks \direct{En sortant.} T'occupe.

\luispeaks Mais qu'est-ce que tu fais ?

\ellespeaks \direct{De la coulisse.} Je cherche quelque chose...

\luispeaks Oh! Oh! Tu vas passer à l'action! Mais la bombe insecticide est dans le placard de la cuisine, ma chérie, pas dans le...

\ellespeaks C'est pas ça que je cherche.

\luispeaks Ah bon? Tu veux l'avoir à la balayette? Non, j'y suis, à la godasse!

\ellespeaks Où est passé ce... machin ?

\luispeaks Je te conseillerais quelque chose de plutôt large, mais maniable, si tu veux l'avoir du premier coup, tiens, mes palmes de plongée par exemple, non, mieux, tes babouches marocaines, c'est élégant et radical. Rien ne vaut la méthode indigène.

\ellespeaks Je préfère ma méthode.

\luispeaks Très bien, je te laisse improviser. D'ailleurs, la chaussure, c'est pas mal, mais c'est salissant: ça va l'écrabouiller, il y aura une tache, et puis...

\ellespeaks Pas grave, les taches...

\luispeaks Quand même... Note que tu as pas tort, les taches c'est rien à côté du corps...Le pire, c'est le corps...

\ellespeaks Ah! J'ai trouvé !

\luispeaks ... avec les pattes, autour, qui continuent à gigot...

\StageDir{Elle entre, un fusil-harpon dans les mains, braqué sur lui. Noir.}

\end{drama}
\end{document}